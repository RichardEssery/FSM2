\documentclass[fleqn]{article}
\usepackage[margin=2cm]{geometry}
\usepackage{amsmath}
\setlength{\mathindent}{0pt}
\usepackage{enumitem}
\usepackage{graphicx}
\usepackage{hyperref}
\usepackage{longtable}
\title{Flexible Snow Model scientific documentation}
\author{Version 2.1.1}
\date{\today}
\begin{document}
\maketitle

{\parindent 0pt
The Flexible Snow Model (FSM2) allows alternative process parametrizations to be combined in a complete model of the mass and energy balances of snow on the ground and in forest canopies. Parametrizations are selected by setting option numbers in a text file before the model is compiled. Parameter values and input data are read from text files when the model is run. Outputs can be written to text or netCDF files. Physical constants, meteorological driving variables, site characteristics, model state variables and parameters are listed in tables 1 to 6; refer to these tables for any variables that are not explicitly defined in the text. Calculations are described below in the order in which they are performed in the code.
}

\section{Constants and variables}

{\parindent0pt

Table 1. Physical constants and quantities assumed to be constant in FSM2 (module {\tt CONSTANTS}).
\begin{longtable}{|l|l|}
\hline
Constant                                           & Value \\
\hline
Heat capacity of air $c_p$                         & 1005 J K$^{-1}$ kg$^{-1}$ \\
Heat capacity of ice $c_{\rm ice}$                 & 2100 J K$^{-1}$ kg$^{-1}$ \\
Heat capacity of water $c_{\rm wat}$               & 4180 J K$^{-1}$ kg$^{-1}$ \\
Saturation vapour pressure at $T_m$, $e_0$         & 611.213 Pa                \\
Acceleration due to gravity $g$                    & 9.81 m s$^{-2}$           \\
Solar constant $I_0$                               & 1367 W m$^{-2}$           \\
Von K\'arm\'an constant $k$                        & 0.4                       \\
Latent heat of fusion of water $L_f$               & $0.334 \times 10^6$ J kg$^{-1}$ \\
Latent heat of sublimation of ice $L_s$            & $2.835 \times 10^6$ J kg$^{-1}$ \\
Latent heat of vapourisation of water $L_v$        & $2.501 \times 10^6$ J kg$^{-1}$ \\
Gas constant for air $R_{\rm air}$                 & 287 J K$^{-1}$ kg$^{-1}$  \\
Gas constant for water vapour $R_{\rm wat}$        & 462 J K$^{-1}$ kg$^{-1}$  \\
Melting point of ice $T_m$                         & 273.15 K                  \\
Ratio of molecular weights of water and dry air $\varepsilon$ & 0.622          \\
Thermal conductivity of air $\lambda_{\rm air}$    & 0.025 W m$^{-1}$ K$^{-1}$ \\
Thermal conductivity of clay $\lambda_{\rm clay}$  & 1.16 W m$^{-1}$ K$^{-1}$  \\
Thermal conductivity of ice $\lambda_{\rm ice}$    & 2.24 W m$^{-1}$ K$^{-1}$  \\
Thermal conductivity of sand $\lambda_{\rm sand}$  & 1.57 W m$^{-1}$ K$^{-1}$  \\
Thermal conductivity of water $\lambda_{\rm wat}$  & 0.56 W m$^{-1}$ K$^{-1}$  \\
Dynamic viscosity of water $\mu_{\rm wat}$         & $1.78 \times 10^{-3}$ kg m$^{-1}$ \\
Density of ice $\rho_{\rm ice}$                    & 917 kg m$^{-3}$           \\
Density of water $\rho_{\rm wat}$                  & 1000 kg m$^{-3}$          \\
Stefan-Boltzmann constant $\sigma$                 & $5.26 \times 10^{-8}$ W m$^{-2}$ K$^{-4}$ \\
\hline 
\end{longtable}

Table 2. Meteorological driving variables.
\begin{longtable}{|l|l|l|}
\hline
Variable                                      & Units \\
\hline
Incoming longwave radiation $LW_\downarrow$   & W m$^{-2}$ \\
Surface air pressure $P_s$                    & Pa \\
Specific humidity $Q_a$                       & kg kg$^{-1}$ \\
Relative humidity $RH$                        & \% \\
Rainfall rate $R_f$                           & kg m$^{-2}$ s$^{-1}$ \\
Snowfall rate $S_f$                           & kg m$^{-2}$ s$^{-1}$ \\
Incoming shortwave radiation $SW_\downarrow$  & W m$^{-2}$ \\
Air temperature $T_a$                         & K \\
Wind speed $U_a$                              & m s$^{-1}$ \\
\hline 
\end{longtable}

Table 3. State variables.
\begin{longtable}{|l|l|l|}
\hline
Variable & Units \\
\hline
\multicolumn{2}{|c|}{Forest canopy ($N_{\rm cnpy}$ layers)} \\
\hline
Canopy air space specific humidity $Q_c$ & kg kg$^{-1}$ \\
Snow mass on canopy $S_v$                & W m$^{-2}$ \\
Canopy air space temperature $T_c$       & K \\
Vegetation temperature $T_v$             & K \\
\hline 
\multicolumn{2}{|c|}{Surface}                \\
\hline 
Surface skin temperature $T_s$           & K \\
\hline 
\multicolumn{2}{|c|}{Snow on the ground (up to $N_{\rm smax}$ layers)}  \\
\hline 
Number of snow layers $N_{\rm snow}$     & - \\
Albedo of snow $\alpha_s$                & - \\
Thickness of snow layers $D_{sn}$        & m \\
Radii of grains in snow layers $r$       & m \\
Ice content of snow layers $I$           & kg m$^{-2}$ \\
Liquid water content of snow layers $W$  & kg m$^{-2}$ \\
Temperature of snow layers $T_{sn}$      & K \\
\hline 
\multicolumn{2}{|c|} {Soil ($N_{\rm soil}$ layers)} \\
\hline 
Temperature of soil layers $T_{sl}$      & K \\
Volumetric moisture content of soil layers $\theta_{sl}$ & - \\
\hline 
\end{longtable}

Table 4. Model layers.
\begin{longtable}{|l|l|l|}
\hline
Documentation                               & Namelist     & Default                \\
\hline
Fraction of vegetation in upper canopy layer $f_\Lambda$ & fvg1 & 0.5               \\ 
Maximum number of snow layers $N_{\rm smax}$  & {\tt Nsmax}  & 3                    \\
Fixed snow layer thicknesses $\Delta z_{sn}$  & {\tt Dzsnow} & 0.1, 0.2, 0.4 m      \\
Number of soil layers $N_{\rm soil}$          & {\tt Nsoil}  & 4                    \\
Soil layer thicknesses $\Delta z_{sl}$        & {\tt Dzsoil} & 0.1, 0.2, 0.4, 0.8 m \\
\hline 
\end{longtable}

Table 5. Site and driving data characteristics.
\begin{longtable}{|l|l|l|}
\hline
Documentation                       & Namelist   & Default         \\
\hline
Snow-free albedo $\alpha_0$         & {\tt alb0} & 0.2             \\
Timestep $\delta t$                 & {\tt dt}   & 3600 s          \\
Latitude $\phi$                     & {\tt lat}  & 0$^\circ$       \\
Local time of solar noon $h_{12}$   & {\tt noon} & 12:00           \\
Vegetation area index $\Lambda$     & {\tt VAI}  & 0               \\
Canopy height $h_c$                 & {\tt vegh} & 0 m             \\
Temperature and humidity measurement height $z_T$& {\tt zT} & 2 m  \\
Wind speed measurement height $z_U$ & {\tt zU}   & 10 m            \\
\hline 
\end{longtable}

Table 6. Parameters (subroutine {\tt FSM2\_PARAMS}).
\begin{longtable}{|l|l|l|}
\hline
Documentation                                      & Namelist   & Default          \\
\hline
Snow-free dense canopy albedo $\alpha_{c0}$        & {\tt acn0} & 0.1              \\
Snow-covered dense canopy albedo $\alpha_{cs}$     & {\tt acns} & 0.3              \\ 
Minimum albedo for melting snow $\alpha_{\rm min}$ & {\tt asmn} & 0.5              \\
Maximum albedo for fresh snow $\alpha_{\rm max}$   & {\tt asmx} & 0.85             \\
Canopy element reflectivity $\alpha_{\Lambda 0}$   & {\tt avg0} & 0.27             \\
Canopy snow reflectivity $\alpha_{\Lambda s}$      & {\tt avgs} & 0.65             \\ 
Vegetation heat capacity per unit VAI $C_\Lambda$  & {\tt cvai} & $3.6 \times 10^4$ J K$^{-1}$ m$^{-2}$ \\
Vegetation turbulent transfer coefficient $C_{\rm veg}$ & {\tt cveg} & 20          \\
Reference snow viscosity $\eta_0$                  & {\tt eta0} & $3.7 \times 10^7$ Pa s \\
Exponential unloading timescale $\tau_u$           & {\tt eunl} & 240 hours        \\
Soil clay fraction $f_{\rm clay}$                  & {\tt fcly} & 0.3              \\
Soil sand fraction $f_{\rm sand}$                  & {\tt fsnd} & 0.6              \\
Surface conductance for saturated soil $g_{\rm sat}$ & {\tt gsat} & 0.01 m s$^{-1}$\\
Surface conductance for snow-free vegetation $g_{\rm veg}$ & {\tt gsnf} & 0.01 m s$^{-1}$   \\
Canopy base height $h_b$                           & {\tt hbas} & 2 m              \\
Snow cover fraction depth scale $h_f$              & {\tt hfsn} & 0.1 m            \\
Canopy light extinction coefficient $k_{\rm ext}$  & {\tt kext} &  0.5             \\
Fixed snow thermal conductivity $\lambda_0$        & {\tt kfix} & 0.24 W m$^{-1}$ K$^{-1}$  \\
Leaf boundary layer resistance $r_{\rm leaf}$      & {\tt leaf} & 20 s$^{1/2}$ m$^{-1/2}$   \\ 
Melt unloading fraction $m_u$                      & {\tt munl} & 0.4                       \\
Number of snow hydrology substeps $N_{\rm shyd}$   & {\tt nhyd} & 10               \\
Maximum density for cold snow $\rho_{\rm cold}$    & {\tt rcld} & 300 kg m$^{-3}$  \\
Fixed snow density $\rho_0$                        & {\tt rfix} & 300 kg m$^{-3}$  \\
Fresh snow grain radius $r_0$                      & {\tt rgr0} & $5\times10^{-5}$ m  \\
Fresh snow density $\rho_f$                        & {\tt rhof} & 100 kg m$^{-3}$  \\
Maximum density for melting snow $\rho_{\rm melt}$ & {\tt rmlt} & 500 kg m$^{-3}$  \\
Snowfall to refresh albedo $S_\alpha$              & {\tt Salb} & 10 kg m$^{-2}$   \\
Thermal metamorphism parameter $c_1$ & {\tt snda}  & $2.8 \times 10^{-6}$ s$^{-1}$ \\
Intercepted snow capacity per unit VAI $S_\Lambda$ & {\tt svai} & 4.4 kg m$^{-2}$  \\
Snow albedo decay temperature threshold $T_\alpha$ & {\tt Talb} & -2$^\circ$C      \\
Cold snow albedo decay time scale $\tau_{\rm cold}$& {\tt tcld} & 1000 hours       \\
Melting snow albedo decay time scale $\tau_{\rm melt}$ & {\tt tmlt} & 100 hours    \\
Snow compaction time scale $\tau_\rho$             & {\tt trho} & 200 hours        \\
Temperature unloading parameter $C_T$              & {\tt Tunl} & $1.87 \times 10^5$ K s  \\
Wind unloading parameter $C_U$                     & {\tt Uunl} & $1.56 \times 10^5$ m    \\
Canopy wind decay coefficient $\eta$               & {\tt wcan} & 2.5              \\
Irreducible liquid water content of snow $W_{\rm irr}$ & {\tt Wirr} & 0.03         \\
Snow-free surface roughness length $z_{0f}$        & {\tt z0sf} & 0.1 m            \\
Snow surface roughness length $z_{0s}$             & {\tt z0sn} & 0.001 m          \\
\hline
\end{longtable}
}

\section{Driving data (subroutine {\tt FSM2\_DRIVE})}

1D driving data are read from a text file with optional formats described in the FSM2 User Guide. A minimum wind speed of 0.1 m s$^{-1}$ is imposed to avoid dividing by small numbers in aerodynamic calculations.

\subsection{FSM format driving data (option {\tt DRIV1D 1})}

Assuming that relative humidity is measured with respect to water at all temperatures, specific humidity is calculated as
\begin{equation}
Q_a = \frac{RH}{100}\frac{\varepsilon e_0}{P_s}\exp\left(\frac{17.5043T_a}{241.3 + T_a}\right)
\end{equation}
for air temperature in $^\circ$C. The humidity is limited to not exceed saturation with respect to ice.

\subsection{ESM-SnowMIP format driving data (option {\tt DRIV1D 2)}}

Specific humidity is included in ESM-SnowMIP driving data files.


\subsection{Shortwave radiation partitioning (option {\tt SWPART 1})}

Subroutine {\tt SOLARPOS} approximates solar declination $\delta$ (radians) and equation of time $E_t$ (hours) by Fourier series
\begin{align}
\delta = 0.006918 &- 0.399912\cos\Gamma  + 0.070257\sin\Gamma  \nonumber \\
                  &- 0.006758\cos2\Gamma + 0.000907\sin2\Gamma \nonumber \\
                  &- 0.002697\cos3\Gamma + 0.001480\sin3\Gamma
\end{align}
and
\begin{equation}
E_t = (12/\pi)(0.000075 + 0.001868\cos\Gamma  - 0.032077\sin\Gamma 
                        - 0.014615\cos2\Gamma - 0.04089\sin2\Gamma)
\end{equation}
for day of year $d_n$ and day angle $\Gamma = 2\pi(d_n - 1)/365$. From a date given by integers $y$, $m$ and $d$ for year, month and day, the day of year can be found using integer division in the magic formula
\begin{equation}
d_n = (7y)/4 - 7(y+(m+9)/12)/4 + (275m)/9 + d - 30.
\end{equation}
For hour $h$ of a day, the hour angle is defined by
\begin{equation}
\omega = (\pi/12)(h_{12} - h - E_t),
\end{equation}
the sine of the solar elevation is
\begin{equation}
\sin\theta = \sin\delta\sin\phi + \cos\delta\cos\phi\cos\omega
\end{equation}
and the cosine of the solar azimuth is
\begin{equation}
\cos\psi = (\sin\theta\sin\phi - \sin\delta)/(\cos\theta\cos\phi).
\end{equation}
Azimuth is measured anticlockwise from south, so $\psi$ has the same sign as $\omega$ (positive in the morning and negative in the afternoon).

An atmospheric clearness parameter is found by dividing global radiation at the surface by incoming radiation at the top of the atmosphere, giving
\begin{equation}
k_t = \frac{SW_\downarrow}{I_0\sin\theta}.
\end{equation}
The diffuse fraction of shortwave radiation is then estimated as
\begin{equation}
\frac{S_{\rm dif}}{SW_\downarrow} = \begin{cases}
    1 - 0.09k_t &  k_t \leq 0.22  \\
    0.95-0.16k_t+4.39k_t^2-16.64k_t^3+12.34k_t^4 &  0.22 < k_t \leq 0.8  \\
    0.165 &  k_t>0.8.
\end{cases}
\end{equation}
Direct-beam shortwave radiation is the remainder $S_{\rm dir} = SW_\downarrow - S_{\rm dif}$.

\section{Forest canopy properties (subroutine {\tt CANOPY})}

A one-layer canopy model (option {\tt CANMOD 1}) has vegetation area index $\Lambda$ provided as an input. The layers in a two-layer model ({\tt CANMOD 2}) have vegetation area indices $\Lambda_1 = f_\Lambda\Lambda$ (upper) and $\Lambda_2 = (1 - f_\Lambda)\Lambda$ (lower). A canopy layer with intercepted snow has heat capacity $C_v = C_\Lambda\Lambda + c_{\rm ice}S_v$, snow interception capacity $S_c = S_\Lambda\Lambda$ and snow cover fraction
\begin{equation}
f_{cs} = \left(\frac{S_v}{S_c}\right)^{2/3}.
\end{equation}
Vegetation layer temperatures at the start of the timestep are saved for use in subroutine {\tt SRFEBAL}.

\section{Shortwave radiation (subroutine {\tt SWRAD})}

\subsection{Ground albedos}

Bare ground with albedo $\alpha_0$ and snow cover fraction $f_s$ with albedo $\alpha_s$ have average albedo
\begin{equation}
\alpha = (1 - f_s)\alpha_0 +f_s\alpha_s.
\end{equation} 
The snow cover fraction for snow depth $h_s$ is
\begin{equation}
f_s = \min\left(\frac{h_s}{h_f},1\right)
\end{equation}
for option {\tt SNFRAC 1},
\begin{equation}
f_s = \tanh\left(\frac{h_s}{h_f}\right)
\end{equation}
for {\tt SNFRAC 2} and
\begin{equation}
f_s = \frac{h_s}{h_s + h_f}
\end{equation}
for {\tt SNFRAC 3}.

\subsubsection{Diagnosed snow albedo (option {\tt ALBEDO 1})}

Snow albedo is diagnosed as a function of surface temperature 
\begin{equation}
\alpha_s = \alpha_{\rm min} + (\alpha_{\rm max} - \alpha_{\rm min})
           \min\left(\frac{T_s - T_m}{T_\alpha}, 1\right).        
\end{equation}

\subsubsection{Prognostic snow albedo (option {\tt ALBEDO 2})}

Snow albedo decreases with time and increases as fresh snow falls with update
\begin{equation}
\alpha_s \rightarrow \alpha_s + (\alpha_{\rm lim} - \alpha_s)(1-e^{-\gamma\delta t}) 
\end{equation}
each timestep, where
\begin{equation}
\gamma = \frac{1}{\tau_\alpha}+\frac{S_f}{S_\alpha}
\end{equation}
and
\begin{equation}
\alpha_{\rm lim} = \frac{1}{\gamma}\left(\frac{1}{\tau_\alpha}\alpha_{\rm min} +\frac{S_f}{S_\alpha}\alpha_{\rm max}\right).
\end{equation}
Timescale $\tau_\alpha$ has different values $\tau_{\rm cold}$ and $\tau_{\rm melt}$ for cold and melting snow.

\subsection{Canopy reflection and transmission (subroutine {\tt SWRAD})}

A canopy layer has reflectivities $R_b$, $R_d$ and transmissivities $\tau_b$, $\tau_d$ for direct-beam and diffuse radiation, respectively, and forward-scattering fraction $s_b$ for direct-beam radiation. Upwards and downwards shortwave radiation fluxes at the top and bottom of canopy layers are found by solving a matrix equation
\begin{equation}
\begin{pmatrix}
1       & -R_d     & 0 \\
-\alpha & 1        & 0 \\
0       & -\tau_d  & 1 \\
\end{pmatrix}
\begin{pmatrix}
S_{\downarrow 1} \\
S_{\uparrow 1}   \\
S_{\uparrow 0}   \\
\end{pmatrix}
=
\begin{pmatrix}
\tau_d  \\
0       \\
R_d     \\
\end{pmatrix}
S_{\downarrow\rm dif} +
\begin{pmatrix}
s_b           \\
\alpha\tau_b  \\
R_b           \\
\end{pmatrix}
S_{\downarrow\rm dir}
\label{eq:canmod1}
\end{equation}
for a one-layer model or
\begin{equation}
\begin{pmatrix}
1           & 0       & 0           & -R_{d,1}    & 0 \\
-\tau_{d,2} & 1       & -R_{d,2}    & 0           & 0 \\
0           & -\alpha & 1           & 0           & 0 \\
-R_{d,2}    & 0       & -\tau_{d,2} & 1           & 0 \\
0           & 0       & 0           & -\tau_{d,1} & 1 \\
\end{pmatrix}
\begin{pmatrix}
S_{\downarrow 1} \\
S_{\downarrow 2} \\
S_{\uparrow 2}   \\
S_{\uparrow 1}   \\
S_{\uparrow 0}   \\
\end{pmatrix}
=
\begin{pmatrix}
\tau_{d,1} \\
0          \\
0          \\
0          \\
R_{d,1}    \\
\end{pmatrix}
S_{\downarrow\rm dif} +
\begin{pmatrix}
s_{b,1}                    \\
s_{b,2}\tau_{b,1}          \\
\alpha\tau_{b,1}\tau_{b,2} \\
R_{b,2}\tau_{b,1}          \\
R_{b,1}                    \\
\end{pmatrix}
S_{\downarrow\rm dir}
\label{eq:canmod2}
\end{equation}
for a two-layer model. Net shortwave radiation absorbed by vegetation in canopy layers and the underlying snow or ground surface are
\begin{gather}
SW_v = S_{\downarrow\rm dif} - S_{\downarrow 1} + S_{\uparrow 1} - S_{\uparrow 0}
         + (1 - \tau_b)S_{\downarrow\rm dir}, \\
SW_s = (1-\alpha)(S_{\downarrow 1} + \tau_b S_{\downarrow\rm dir})
\end{gather}
for a one-layer model, and
\begin{gather}
SW_{v,1} = S_{\downarrow\rm dif} - S_{\downarrow 1} + S_{\uparrow 1} - S_{\uparrow 0} +
          (1 - \tau_{b,1})S_{\downarrow\rm dir}, \\
SW_{v,2} = S_{\downarrow 1} - S_{\downarrow 2} + S_{\uparrow 2} - S_{\uparrow 1} +
          \tau_{b,1}(1 - \tau_{b,2})S_{\downarrow\rm dir}, \\
SW_s = (1-\alpha)(S_{\downarrow 2} + \tau_{b,1}\tau_{b,2}S_{\downarrow\rm dir})
\end{gather}
for a two-layer model. Equations (\ref{eq:canmod1}) and (\ref{eq:canmod2}) are solved by LU decomposition (subroutine {\tt LUDCMP}).

\subsubsection{Beer's Law (option {\tt CANRAD 1})}

The fractions of radiation transmitted without interception through a canopy layer are
\begin{equation}
\tau_b = \exp(-k_{\rm ext}\Lambda/\sin\theta)
\end{equation}
and
\begin{equation}
\tau_d = \exp(-1.6k_{\rm ext}\Lambda).
\end{equation}
Diffuse and direct-beam canopy layer reflectivities are $R_d = (1-\tau_d)\alpha_c$ and $R_b = (1-\tau_b)\alpha_c$ for dense canopy albedo
\begin{equation}
\alpha_c = (1 - f_{cs})\alpha_{c0} + f_{cs}\alpha_{cs}
\end{equation}
and the forward-scattering fraction is zero.

\subsubsection{Two-stream approximation (option {\tt CANRAD 2}, subroutine {\tt TWOSTREAM})}

For flat, opaque and randomly oriented leaves, 
\begin{equation}
\omega = (1 - f_{cs})\alpha_{\Lambda 0} + f_{cs}\alpha_{\Lambda s}
\end{equation}
is the fraction of incident radiation that is scattered, $\beta = 2/3$ is the fraction of scattered diffuse radiation that is directed back into the upward hemisphere and
\begin{equation}
\beta_0 = (0.5 + \mu)\left[1 - \mu\ln\left(\frac{1 + \mu}{\mu}\right)\right]
\end{equation}
is the upscatter fraction for direct-beam radiation with $\mu = \sin\theta$. Two-stream equations involve coefficients
\begin{equation}
\gamma_1 = 2[1 - (1 - \beta)\omega],\ \gamma_2 = 2\beta\omega,\ \gamma_3 = \beta_0,\ \gamma_4 = 1 - \beta_0.
\end{equation}
The reflectivity and transmittivity of a canopy layer for diffuse radiation are
\begin{equation}
R_d = \frac{\gamma_2(1 - e^{-2kl})}{k + \gamma_1 + (k - \gamma_1)e^{-2kl}}
\label{eq:rdif}
\end{equation}
and
\begin{equation}
\tau_d = \frac{2ke^{-kl}}{k + \gamma_1 + (k - \gamma_1)e^{-2kl}}
\end{equation}
for extinction coefficient $k = (\gamma_1^2 - \gamma_2^2)^{1/2}$ and optical thickness $l = k_{\rm ext}\Lambda$. The direct-beam reflectivity, forward scattering fraction and transmissivity are
\begin{equation}
R_b = \frac
{\omega[(1 - k\mu)(\alpha_2 + k\gamma_3)e^{kl} - (1 + k\mu)(\alpha_2 - k\gamma_3)e^{-kl} - 2k(\gamma_3 - \alpha_2\mu)e^{-l/\mu}]} {(1 - k^2\mu^2)[(k + \gamma_1)e^{kl} + (k - \gamma_1)e^{-kl}]}
\end{equation}
and
\begin{equation}
s_b = \frac{\omega e^{-l/\mu}[(1 - k\mu)(\alpha_1 - k\gamma_4)e^{-kl} - (1 + k\mu)(\alpha_1 + k\gamma_4)e^{kl}] + 2k\omega(\gamma_4 + \alpha_1\mu)}
{(1 - k^2\mu^2)[(k + \gamma_1)e^{kl} + (k - \gamma_1)e^{-kl}]},
\end{equation}
and
\begin{equation}
\tau_b = e^{-l/\mu}
\end{equation}
where $\alpha_1 = \gamma_1\gamma_4 + \gamma_2\gamma_3$ and $\alpha_2 = \gamma_1\gamma_3 + \gamma_2\gamma_4$.

\section{Thermal properties (subroutine {\tt THERMAL})}

\subsection{Snow}
The thermal conductivity of a snow layer of density $\rho_s$ is a fixed parameter for option {\tt CONDCT 0} and 
\begin{equation}
\lambda_{sn} = 2.224\left(\frac{\rho_s}{\rho_{\rm wat}}\right)^{1.885}
\end{equation}
for {\tt CONDCT 1}. 

\subsection{Soil}
Clapp-Hornberger exponent
\begin{equation}
b = 3.1 + 15.7f_{\rm clay} - 0.3f_{\rm sand},
\end{equation}
dry soil heat capacity
\begin{equation}
c_{\rm dry} = \frac{2.128\times10^6f_{\rm clay} + 2.385\times10^6f_{\rm sand}}{f_{\rm clay} + f_{\rm sand}},
\end{equation}
saturated soil water suction
\begin{equation}
\Psi_s = 10^{0.17 - 0.63f_{\rm clay} - 1.58f_{\rm sand}},
\end{equation}
volumetric soil moisture at saturation
\begin{equation}
V_{\rm sat} = 0.505 - 0.037f_{\rm clay} - 0.142f_{\rm sand},
\end{equation}
volumetric soil moisture at the critical point
\begin{equation}
V_{\rm crit} = V_{\rm sat}\left(\frac{\Psi_s}{3.364}\right)^{1/b}
\end{equation}
and dry soil thermal conductivity
\begin{equation}
\lambda_{\rm dry} = \lambda_{\rm air}^{V_{\rm sat}}(\lambda_{\rm clay}^{f_{\rm clay}}\lambda_{\rm sand}^{1 - f_{\rm clay}})^{1 - V_{\rm sat}}
\end{equation}
are derived from soil texture fractions $f_{\rm clay}$ and $f_{\rm sand}$ at the start of the run.

The temperature derivative of soil water suction $\Psi$ in the presence of ice is
\begin{equation}
\frac{d\Psi}{dT} = - \frac{\rho_{\rm ice}L_f}{g\rho_{\rm wat}T_m}.
\end{equation}
For a layer with volumetric soil moisture content $V_s$, the temperature above which all soil moisture is unfrozen is 
\begin{equation}
T_{\rm max} = T_m + \Psi_s\left(\frac{d\Psi}{dT}\right)^{-1}\left(\frac{V_{\rm sat}}{V_s}\right)^b.
\end{equation}
The apparent areal heat capacity of a soil layer of thickness $\Delta z_{\rm sl}$ and Celsius temperature $T_c = T_{\rm sl} - T_m$, including the influence of soil moisture phase change, is
\begin{equation}
C_{sl} = (c_{\rm dry} + \rho_{\rm ice}c_{\rm ice}\theta_f + \rho_{\rm wat}c_{\rm wat}\theta_u)\Delta z_{\rm sl} +      
               \rho_{\rm wat}\Delta z_{\rm sl}[(c_{\rm wat} - c_{\rm ice})T_c + L_f]\frac{d\theta_u}{dT}
\end{equation}
for
\begin{equation}
\frac{d\theta_u}{dT} = - \frac{d\Psi}{dT}\frac{V_{\rm sat}}{b\Psi_s}
                         \left(\frac{T_c}{\Psi_s}\frac{d\Psi}{dT}\right)^{-1/b-1}.
\end{equation}
The thermal conductivity of wet soil is
\begin{equation}
\lambda_{sl} = (\lambda_{\rm sat} - \lambda_{\rm dry})(S_f + S_u) + \lambda_{\rm dry}
\end{equation}
with
\begin{equation}
\lambda_{\rm sat} = \frac{\lambda_{\rm soil}\lambda_{\rm wat}^{\theta_{\rm wat}}\lambda_{\rm ice}^{\theta_{\rm ice}}}        
                         {\lambda_{\rm air}^{V_{\rm sat}}}
\end{equation}
for
\begin{equation}
\theta_{\rm ice} = \frac{S_f}{S_f + S_u}V_{\rm sat},\quad
\theta_{\rm wat} = \frac{S_u}{S_f + S_u}V_{\rm sat}.
\end{equation}

\subsection{Surface layer}
Surface fluxes are calculated using the thermal properties of a surface layer of thickness $\Delta z_1 = \max(\Delta z_{sl,1},D_{sn,1})$ that can include both snow and soil for shallow snow. The average temperature of this layer is 
\begin{equation}
T_1 = 
\begin{cases}
      T_{sl,1} + (T_{sn,1} - T_{sl,1})\frac{D_{sn,1}}{\Delta z_{sl,1}}  & h_s \leq \Delta z_{sl,1} \\
      T_{sn,1}                                                & h_s > \Delta z_{sl,1} \\
\end{cases}
\end{equation}
and the thermal conductivity between the surface and the middle of the layer if snow free is
\begin{equation}
\lambda_1 = 
\begin{cases}
      \Delta z_{sl,1}\left(\frac{2D_{sn,1}}{\lambda_{sn,1}} + \frac{\Delta z_{sl,1} - 2D_{sn,1}}{\lambda_{sl,1}}\right)^{-1}  & h_s \leq 0.5\Delta z_{sl,1} \\
      \lambda_{sn}                                                & h_s > 0.5\Delta z_{sl,1} \\
\end{cases}
\end{equation}
The moisture conductance for evaporation between this layer and the atmosphere is
\begin{equation}
g_1 = g_{\rm sat}\left(\frac{\theta_u}{V_{\rm crit}}\right)^2.
\end{equation}

\section{Surface and canopy energy balance (subroutine {\tt SRFEBAL})}

Quantities set before starting the iterative solution of the surface energy balance are:
\begin{itemize}[leftmargin=*]
\item temperature and wind speed measurement heights $z_T$ and $z_U$, incremented by $h_c$ if specified above the canopy height (option {\tt ZOFFST 1});
\item canopy layer heights $z_1 = h_b + (h_c-h_b)/2$ (one layer) or $z_1 = (1 - f_\Lambda/2)h_c$ and $z_2 = (1 - f_\Lambda)h_c/2$ (two layers);
\item vegetation fraction $f_v = 1 - \exp(-k_{\rm ext}\Lambda)$;
\item vegetation roughness length $z_{0v} = 0.1h_c$ and displacement height $d = 0.67h_c$; 
\item momentum roughness length $z_0 = z_{0s}^{f_s}z_{0f}^{1-f_s}$;
\item scalar roughness length $z_{0h} = 0.1z_0$;
\item air density $P_s/(R_{\rm air}T_a)$.
\end{itemize}
Saturation humidity (function {\tt qsat}) and latent heat
\begin{equation}
q_{\rm sat}(T,P_s) = 
\begin{cases}
     \frac{\varepsilon e_0}{P_s}\exp\left(\frac{22.4422T_a}{272.186 + T_a}\right), \quad L = L_s   & T \leq T_m \\
     \frac{\varepsilon e_0}{P_s}\exp\left(\frac{17.5043T_a}{241.3 + T_a}\right), \quad L = L_v  & T > T_m  \\
\end{cases}
\end{equation}
and gradient 
\begin{equation}
D = \frac{dq_{\rm sat}}{dT} = \frac{Lq_{\rm sat}}{R_{\rm wat}T^2}
\end{equation}
are set wherever required below for surface or vegetation temperatures.

\subsection{Open areas ($\Lambda = 0$)}

The reciprocal of the Obukhov length, the friction velocity and the aerodynamic conductance are
\begin{gather}
\frac{1}{L_O} = - \frac{kgg_a(T_s - T_a)}{T_au_*^3}, \\
u_* = kU_a\left[\ln\left(\frac{z_U}{z_0}\right) -
                \psi_m\left(\frac{z_U}{L_O}\right) + \psi_m\left(\frac{z_0}{L_O}\right)\right]^{-1}
\end{gather}
and
\begin{equation}
g_a = ku_*\left[\ln\left(\frac{z_T}{z_{0h}}\right) -
                \psi_h\left(\frac{z_T}{L_O}\right) + \psi_h\left(\frac{z_{0h}}{L_O}\right)\right]^{-1},
\end{equation}
with $\psi_h$ and $\psi_m$ set to zero for option {\tt EXCHNG 0} and calculated by functions {\tt psih} and {\tt psim} for {\tt EXCHNG 1}. The surface miosture flux, ground heat flux, sensible heat flux and net radiation are 
\begin{gather}
E = \rho\chi_sg_a[q_{\rm sat}(T_s) - q_a], \\
G = \frac{2\lambda_1}{\Delta z_1}(T_s - T_1), \\
H = \rho c_pg_a(T_s - T_a), \\
R_s = SW_s + LW_\downarrow - \sigma T_s^4,
\end{gather}
with water availability factor $\chi_s = 1$ if $q_a > q_{\rm sat}$ and
\begin{equation}
\chi_s = f_s + \frac{(1 - f_s)g_1}{g_a + g_1}
\end{equation}
otherwise. The surface temperature and flux increments are first calculated assuming no snow melt ($M = 0$) as
\begin{gather}
\delta T_s = \frac{R_s - G - H - LE - L_fM}{4\sigma T_s^3 + 2\lambda_1/\Delta z_1 + \rho g_a(c_p + LD\chi_s)}, \\
\delta E = \rho\chi_sg_aD\delta T_s, \\
\delta G = \frac{2\lambda_1}{\Delta z_1}\delta T_s
\end{gather}
and
\begin{equation}
\delta H = \rho c_pg_a\delta T_s.
\end{equation}
If this gives a temperature $T_s + \delta T_s > T_m$ and there is snow with ice mass $I$ on the ground, the temperature and flux increments are recalculated with $M = I / \delta t$ (all of the snow melts). If then $T_s + \delta T_s < T_m$, not all of the snow melts and the fluxes are recalculated with $T_s = T_m$ and the melt rate is diagnosed as
\begin{equation}
M =  (R_s - G - H - LE)/L_f.
\end{equation}
The increments are added to the surface temperature and fluxes, and the surface energy balance residual
\begin{equation}
\epsilon =  R_s - G - H - LE - L_fM
\end{equation}
is calculated. At least four iterations and at most ten are made, terminating earlier if $|\epsilon | < 0.01$ W m$^{-2}$.

\subsection{Forest ($\Lambda > 0$)}

Aerodynamic quantities for forests are calculated as weighted averages between open and dense-canopy values, according to the vegetation fraction. The friction velocity, aerodynamic conductance between the highest canopy layer and the atmosphere, and aerodynamic conductance between the lowest canopy layer and the ground for a dense canopy are
\begin{gather}
u_{*d} = kU_a\left[\ln\left(\frac{z_U-d}{z_{0v}}\right) - \psi_m\left(\frac{z_U-d}{L_O}\right) + \psi_m\left(\frac{z_{0v}}{L_O}\right)\right]^{-1}, \\
g_{ad} = \left\{\frac{1}{ku_*}\left[\ln\left(\frac{z_T-d}{h_c-d}\right) 
      - \psi_H\left(\frac{z_T-d}{L_O}\right) + \psi_H\left(\frac{h_c-d}{L_O}\right)\right]
      + \frac{h_c\left[e^{\eta(1-z_1/h_c)}-1\right]}{\eta K_H}\right\}^{-1}
\end{gather}
and
\begin{equation}
g_{sd} = \left[\frac{1}{k^2U_b}\ln\left(\frac{h_b}{z_0}\right)\ln\left(\frac{h_b}{z_{0h}}\right) +
       \frac{e^\eta h_c}{\eta K_H}\left(e^{-\eta h_b/h_c} - e^{-\eta z_N/h_c}\right)\right]^{-1}
\end{equation}
with
\begin{equation}
K_H = 
\begin{cases}
    ku_*(h_c-d)/[1 + 5(h_c -d)/L_O]        & L_O > 0 \\
    ku_*(h_c-d)[1 - 16(h_c - d)/L_O]^{1/2} &  L_O < 0.
\end{cases}
\end{equation}
Conductance between a vegetation layer and its corresponding canopy air space is 
\begin{equation}
g_v = \frac{U_c^{1/2}\Lambda_n}{r_{\rm leaf}}
\end{equation}
for canopy wind speed
\begin{equation}
U_c = f_v\exp\left[\eta\left(\frac{z_n}{h_c} - 1\right)\right]U_h  + 
      (1 - f_v)\frac{u_*}{k}\left[\ln\left(\frac{z_n}{z_0}\right) - \psi_m\left(\frac{z_n}{L_O}\right) + \psi_m\left(\frac{z_0}{L_O}\right)\right]
\end{equation}
and canopy-top wind speed
\begin{equation}
U_h = \frac{u_*}{k}\left[\ln\left(\frac{h_c-d}{z_{0v}}\right) - \psi_m\left(\frac{h_c-d}{L_O}\right) + \psi_m\left(\frac{z_{0v}}{L_O}\right)\right].
\end{equation}
If there are two canopy layers, the conductance between the air spaces is
\begin{equation}
g_c = \frac{f_v\eta K_H}{e^\eta h_c}\left(e^{-\eta z_2/h_c} - e^{-\eta z_1/h_c}\right)^{-1} +
      (1 - f_v)ku_*\left[\ln\left(\frac{z_1}{z_2}\right) - \psi_m\left(\frac{z_1}{L_O}\right) + \psi_m\left(\frac{z_2}{L_O}\right)\right]^{-1}.
\end{equation}
The water availability factor for a canopy layer with snow cover fraction $f_{cs}$ is $\chi_v = 1$ if $q_c > q_{\rm sat}(T_v)$ and
\begin{equation}
\chi_v = f_{cs} + \frac{(1 - f_{cs})g_{\rm veg}}{g_v + g_{\rm veg}}
\end{equation}
otherwise.

\subsubsection{One-layer canopy (option {\tt CANMOD 1})}

Sensible heat fluxes are
\begin{equation}
H = \rho c_p g_a(T_c - T_a)
\end{equation}
from the canopy air space to the atmosphere,
\begin{equation}
H_v = \rho c_p g_v(T_v - T_c)
\end{equation}
from vegetation to the canopy air space, and
\begin{equation}
H_s = \rho c_p g_s(T_s - T_c)
\end{equation}
from the snow or ground surface to the canopy air space. The corresponding moisture fluxes are
\begin{gather}
E = \rho g_a(q_c - q_a), \\
E_v = \chi_v\rho g_v[q_{\rm sat}(T_v) - q_c],
\end{gather}
and
\begin{equation}
E_s = \chi_s\rho g_s[q_{\rm sat}(T_s) - q_c].
\end{equation}
Net radiation absorbed by the surface and the vegetation are
\begin{equation}
R_s = SW_s + \tau_d LW_\downarrow - \sigma T_s^4 + (1-\tau_d)\sigma T_v^4 
\end{equation}
and
\begin{equation}
R_v = SW_v + (1 - \tau_d)(LW_\downarrow + 4\sigma T_s^4 - 2\sigma T_v^4).
\end{equation}
State variable increments in a vector
\begin{equation}
{\bf \delta x} = (\delta T_s, \delta Q_c, \delta T_c, \delta T_v)
\end{equation} 
are found by using LU decomposition to solve matrix equation ${\rm J}{\bf \delta x} = {\bf f}$ with Jacobian matrix elements
\begin{gather}
J_{11} = -\rho g_s(c_p+LD\chi_s) - 4\sigma T_s^3 - 2k_1/\Delta z_1, \quad
J_{12} = L\rho\chi_s g_s, \quad J_{13} = \rho c_p g_s, \quad J_{14} = 4(1 - \tau_d)\sigma T_v^3, \nonumber \\
J_{21} = 4(1 - \tau_d)\sigma T_s^3, \quad J_{22} = L\rho\chi_v g_v, \quad J_{23} = \rho c_p g_v, \nonumber \quad
J_{24} = -\rho g_v(c_p+LD\chi_v) - 8(1-\tau_d)\sigma T_v^3 - C_v \delta t, \nonumber \\
J_{31} = -g_s, \quad J_{32} = 0, \quad J_{33} = g_a+g_s+g_v, \quad J_{34} = -g_v, \nonumber \\
J_{41} = -D\chi_s g _s, \quad J_{42} = g_a +\chi_s g_s + \chi_v g_v, \quad J_{43} = 0, \quad J_{44} = -D\chi_v g_v
\end{gather}
and flux conservation vector elements
\begin{gather}
f_1 = -(R_s - G_s - H_s - LE_s), \nonumber \\
f_2 = -[R_v - H_s - LE_s - C_v(T_v - T_v^{(0)})/\delta t], \nonumber \\
f_3 = -(H - H_s - H_v)/(\rho c_p), \nonumber \\
f_4 = -(E - E_s - E_v)/\rho.
\end{gather}
The flux increments are then
\begin{gather}
\delta E_s = \rho \chi_s g_s(D\delta T_s - \delta Q_c), \nonumber \\
\delta E_v = \rho \chi_v g_v(D\delta T_v - \delta Q_c), \nonumber \\
\delta G_s = 2 k_1\delta T_s / \Delta z_1, \nonumber \\
\delta H_s = \rho c_p g_s(\delta T_s - \delta T_c), \nonumber \\
\delta H_v = \rho c_p g_v(\delta T_v - \delta T_c).
\end{gather}

\subsubsection{Two-layer canopy (option {\tt CANMOD 2})}

Sensible heat fluxes are
\begin{equation}
H = \rho c_p g_a(T_{c,1} - T_a)
\end{equation}
from the canopy air space to the atmosphere,
\begin{equation}
H_{v,n} = \rho c_p g_{v,n}(T_{v,n} - T_{c,n})
\end{equation}
from vegetation layer $n$ to canopy air space layer $n$, and
\begin{equation}
H_s = \rho c_p r_s(T_s - T_{c,2})
\end{equation}
from the ground to the lower canopy air space. The flux between the canopy air space layers is
\begin{equation}
H_c = \rho c_p g_c(T_{c,2} - T_{c,1}).
\end{equation}
The corresponding moisture fluxes are
\begin{gather}
E = \rho g_a(q_{c,1} - q_a), \\
E_{v,n} = \chi_{v,n}\rho g_{v,n}[q_{\rm sat}(T_{v,n}) - q_{c,n}], \\
E_s = \chi_s\rho g_s[q_{\rm sat}(T_s) - q_{c,2}]
\end{gather}
and
\begin{equation}
E_c = \rho g_c(q_{c,2} - q_{c,1}).
\end{equation}
Net radiation absorbed by the surface and the vegetation layers are
\begin{gather}
R_s = SW_s + \tau_{d,1}\tau_{d,2}LW_\downarrow + (1-\tau_{d,1})\tau_{d,2}\sigma T_{v,1}^4 + (1-\tau_{d,2})\sigma T_{v,2}^4 
           - \sigma T_s^4, \\
R_{v,1} = SW_{v,1} + (1 - \tau_{d,1})[LW_\downarrow - 2\sigma T_{v,1}^4 + (1-\tau_{d,2})\sigma T_{v,2}^4 + \tau_{d,2}\sigma T_s^4]
\end{gather}
and
\begin{equation}
R_{v,2} = SW_{v,2} + (1 - \tau_{d,2})[\tau_{d,1}LW_\downarrow + (1 - \tau_{d,1})\sigma T_{v,1}^4 - 2\sigma T_{v,2}^4 + \sigma T_s^4].
\end{equation}
State variable increments in a vector
\begin{equation}
{\bf \delta x} = (\delta T_s, \delta Q_{c,1}, \delta T_{c,1}, \delta T_{v,1}, \delta Q_{c,2}, \delta T_{c,2}, \delta T_{v,2})
\end{equation} 
are found by using LU decomposition to solve matrix equation ${\rm J}{\bf \delta x} = {\bf f}$ with non-zero Jacobian matrix elements  
\begin{gather}
J_{11} = -(c_p + LD\psi_g)\rho g_s - 4\sigma T_s^3 - 2\lambda_1/\Delta z_1, \quad
J_{14} = 4(1 - \tau_{v,1})\tau_{v,2}\sigma T_{v,1}^3, \nonumber \\
J_{15} = L\rho\psi_g g_g, \quad
J_{16} = \rho c_p g_g, \quad
J_{17} = 4(1 - \tau_{v,2})\sigma T_{v,2}^3, \nonumber \\
J_{21} = 4(1 - \tau_{v,1})\tau_{v,2}\sigma T_{v,2}^3, \quad
J_{22} = L\rho\chi_{v,1}g_{v,1}, \quad
J_{23} = \rho c_p g_{v,1}, \nonumber \\
J_{24} = - \rho g_{v,1}(c_p + LD\chi_{v,1}) - 8(1 - \tau_{v,1})\sigma T_{v,1}^3 - C_{v,1}/\delta t, \quad
J_{27} = 4(1 - \tau_{v,1})(1 - \tau_{v,2})\sigma T_{v,2}^3, \nonumber \\
J_{31} = 4(1 - \tau_{v,2})\sigma T_s^3, \quad
J_{34} = 4(1 - \tau_{v,1})(1 - \tau_{v,2})\sigma T_s^3, \nonumber\\
J_{35} = L\rho\chi_{v,2}g_{v,2}, \quad
J_{36} = \rho c_p g_{v,2}, \quad
J_{37} = - \rho g_{v,2}(c_p + LD\chi_{v,2}) - 8(1 - \tau_{v,2})\sigma T_{v,2}^3 - C_{v,2}/\delta t, \nonumber \\
J_{43} = g_a + g_c + g_{v,1}, \quad
J_{44} = -g_{v,1}, \quad
J_{46} = -g_c, \nonumber \\
J_{51} = -g_s, \quad
J_{53} = -g_c, \quad
J_{56} = g_c + g_s + g_{v,2}, \quad
J_{57} = -g_{v,2}, \nonumber \\
J_{62} = g_a + g_c + \chi_{v,1}g_{v,1}, \quad
J_{64} = -D\chi_{v,1}g_{v,1}, \quad
J_{65} = -g_c, \nonumber \\
J_{71} = -D\chi_s g_s, \quad
J_{72} = -g_c, \quad
J_{75} = g_c + \chi_s g_s + \chi_{v,2}g_{v,2}, \quad
J_{77} = -D\chi_{v,2}g_{v,2},
\end{gather}
the flux conservation vector elements are
\begin{gather}
f_1 = -(R_s - G_s - H_s - LE_s), \nonumber \\
f_2 = -[R_{v,1} - H_{v,1} - LE_{v,1} - C_{v,1}(T_{v,1} - T_{v,1}^{(0)})/\delta t], \nonumber \\
f_3 = -[R_{v,2} - H_{v,2} - LE_{v,2} - C_{v,2}(T_{v,2} - T_{v,2}^{(0)})/\delta t], \nonumber \\
f_4 = -(H - H_c - H_{v,1})/(\rho c_p), \nonumber \\
f_5 = -(H_c - H_s - H_{v,2})/(\rho c_p), \nonumber \\
f_6 = -(E - E_c - E_{v,1})/\rho, \nonumber \\
f_7 = -(E_c - E_s - E_{v,2})/\rho
\end{gather}  
and the flux increments are
\begin{gather}
\delta Es = \rho \chi_s g_s(D\delta Ts - \delta Q_{c,2}), \nonumber \\
\delta E_{v,1} = \rho \chi_{v,1} g_{v,1} (D\delta T_{v,1} - \delta Q_{c,1}), \nonumber \\
\delta E_{v,2} = \rho \chi_{v,2} g_{v,2} (D\delta T_{v,2} - \delta Q_{c,2}), \nonumber \\
\delta G_s = 2 k_1 \delta T_s/\Delta z_1, \nonumber \\
\delta H_s = \rho c_p g_s(\delta Ts - \delta T_{c,2}), \nonumber \\
\delta H_{v,1} = \rho c_p g_{v,1}(\delta T_{v,1} - \delta T_{c,1}), \nonumber \\
\delta H_{v,2} = \rho c_p g_{v,2}(\delta T_{v,2} - \delta T_{c,2}).
\end{gather}  

\subsubsection{Sub-canopy snow melt}
If the updated surface temperature exceeds $T_m$ and there is snow with ice mass $I$ on the ground, it is assumed to melt at rate $M=I/\delta t$, melt energy $L_f M$ is added to $f_1$ and the increments are recalculated. If the updated surface temperature is then less than $T_m$, the snow does not all melt in the timestep. The surface fluxes and ${\bf f}$ are recalculated with $T_s = T_m$, elements in the first row of the Jacobian matrix are changed to 0 except $J_{11}=-1$ and the increments are recalculated. The melt rate is then given by $M = \delta x_1/L_f$.


\subsubsection{Sub-canopy diagnostics}
Sub-canopy windspeed at height $z$ is diagnosed as
\begin{equation}
U = f_vU_b\frac{\ln(z/z_{0g})}{\ln(h_b/z_{0g})} + 
    (1-f_v)U_a\left[\frac{\ln(z/z_{0g}) - \psi_m(z/L_O) + \psi_m(z_{0g}/L_O)}
                         {\ln(z_U/z_{0g}) - \psi_m(z_U/L_O) + \psi_m(z_{0g}/L_O)}\right]
\end{equation}
for canopy-base wind speed
\begin{equation}
U_b = \exp\left[\eta\left(\frac{h_b}{h_c}-1\right)\right].
\end{equation}
The sub-canopy air temperature is
\begin{equation}
T = T_s - \frac{H_s}{\rho c_p g_s}
\end{equation}
for surface resistance
\begin{equation}
g_s = \frac{f_vk^2U_b}{\ln(z/z_{0g})\ln(z/z_{0h})} +
      \frac{(1-f_v)ku_*}{\ln(z/z_{0h}) - \psi_h(z/L_O) + \psi_h(z_{0h}/L_O)}.
\end{equation}

\subsection{Stability functions (functions {\tt psim}, {\tt psih})}
Arguments $z$ and $L_O$ define $\zeta = z/L_O$. The momentum and scalar stability functions are
\begin{equation}
\psi_m(\zeta) = 
\begin{cases}
    2\ln\left(\frac{1+x}{2}\right) + \ln\left(\frac{1+x^2}{2}\right) 
    - 2\arctan x + \frac{\pi}{2}    & \zeta<0 \\
    -5\zeta                          &  0 \leq \zeta \\
\end{cases}
\end{equation}
and
\begin{equation}
\psi_H(\zeta) = 
\begin{cases}
    2\ln\left(\frac{1+x^2}{2}\right) & \zeta<0 \\
    -5\zeta                          &  0 \leq \zeta, \\
\end{cases}
\end{equation}
with $x = (1 - 16\zeta)^{1/4}$ and $-2 \leq\zeta\leq 1$. 

\section{Canopy snow mass balance (subroutine {\tt INTERCEPT})}

The mass of snow added to a canopy layer by interception of snowfall in a timestep is either
\begin{equation}
\delta S_v = f_vS_f\delta t
\end{equation}
for linear interception option {\tt CANINT 1} or
\begin{equation}
\delta S_v = (S_c - S_v)\left[1 - \exp\left(-\frac{f_vS_f\delta t}{S_c}\right)\right]
\end{equation}
for non-linear interception option {\tt CANINT 2}, with the limitation $\delta S_v \leq S_c - S_v$.
The rate of snowfall reaching the next layer or the ground is reduced to $S_f - \delta S_v/\delta t$. Snow sublimates from a canopy layer at rate $E_v$ if $E_v>0$ or is added to the canopy layer snow mass if $E_v<0$ and $T_v<T_m$. If the vegetation layer temperature is greater than $T_m$, the amount of melt is $M = L_f^{-1}C_v(T_v - T_m)$, $T_v$ is reset to $T_m$, and the melt water drips from the canopy to the ground. The mass of snow removed by unloading in a timestep is
\begin{equation}
\delta S_v = \frac{\delta t}{\tau_u}S_v + m_uM
\end{equation}
for time/melt-dependent unloading ({\tt CANUNL 1}) or
\begin{equation}
\delta S_v = \left[\frac{1}{c_T}\max(T_v-270.15, 0) + \frac{U_a}{c_U}\right]S_v,
\end{equation}
for temperature/wind-dependent unloading ({\tt CANUNL 2}).

\section{Snow on the ground (subroutine {\tt SNOW})}

\subsection{Heat conduction}

The heat capacity of a snow layer with ice content $I$ and liquid water content $W$ is
\begin{equation}
C_{sn} = c_{\rm ice}I + c_{\rm wat}W. 
\end{equation}
If there is one snow layer, the increment to the layer temperature due to heat conduction over a timestep is
\begin{equation}
\delta T_{sn,1} = \frac{[G_s - U_1(T_{sn,1} - T_{sl,1})]\delta t}{C_{sn,1} + U_1\delta t}
\end{equation}
for thermal transmittance
\begin{equation}
U_1 = 2\left(\frac{D_{sn,1}}{\lambda_{sn,1}} + \frac{\Delta z_{sl,1}}{\lambda_{sl,1}}\right)^{-1}.
\end{equation}
If there is more than one snow layer, snow layer temperature increments are found by solving a tridiagonal matrix equation (subroutine {\tt TRIDIAG}) with thermal transmittance between layers ($U$), below-diagonal ($a$), diagonal ($b$) and above-diagonal ($c$) matrix elements, and right-hand side vector elements ($r$)
\begin{align}
U_1 &= 2\left(\frac{D_{sn,1}}{\lambda_{sn,1}} + \frac{D_{sn,2}}{\lambda_{sn,2}}\right)^{-1} \\
b_1 &= C_{sn,1} + U_1 \delta t \\
c_1 &= -U_1\delta t \\
r_1 &=  [G_s - U_1(T_{sn,1} - T_{sn,2})]\delta t
\end{align}
for $n=1$,
\begin{align}
U_n &= 2\left(\frac{D_{sn,n}}{\lambda_{sn,n}} + \frac{D_{sn,n+1}}{\lambda_{sn,n+1}}\right)^{-1} \\
a_n &= c_{n-1} \\
b_n &= C_{sn,n} + (U_{n-1} + U_n)\delta t \\
c_n &= -U_n\delta t \\
r_n &= U_{n-1}(T_{sn,n-1} - T_{sn,n})\delta t + U_n(T_{sn,n+1} - T_{sn,n})\delta t
\end{align}
for $n=2,...,N_{\rm snow}-1$, and
\begin{align}
U_n &= 2\left(\frac{D_{sn,n}}{\lambda_{sn,n}} + \frac{\Delta z_{sl,1}}{\lambda_{sl,1}}\right)^{-1} \\
a_n &= c_{n-1} \\
b_n &= C_{sn,n} + (U_{n-1} + U_n)\delta t \\
r_n &= U_{n-1}(T_{sn,n-1} - T_{sn,n})\delta t + U_n(T_{sl,1} - T_{sn,n})\delta t
\end{align}
for $n=N_{\rm snow}$. The heat flux at the base of the snow into the soil is $G_{\rm soil} = U_{N_{\rm snow}}(T_{sn,N_{\rm snow}} - T_{sl,1})$.

\subsection{Melt}
The total amount of ice removed by surface melting is $\delta I = M\delta t$. If heat conduction without phase change results in a snow layer temperature $T_{sn} > T_m$, layer melt $L_f^{-1}C_{sn}(T_{sn} - T_m)$ is added to $\delta I$ and the temperature is reset to $T_m$. Starting from the top layer and working downwards, a layer melts entirely if $\delta I$ exceeds the layer ice mass; the ice in the layer is then entirely turned to liquid water, the layer is removed by setting its thickness to zero, and the ice mass is subtracted from $\delta I$ for the next layer. If the layer only partially melts, the melt is added to the liquid content of the layer, the layer thickness is reduced to $(1 - \delta I/I)D_{sn}$ and the loop through layers terminates. 

\subsection{Sublimation}
The total amount of ice removed by surface sublimation is $\delta I = E\delta t$. Starting from the top layer and working downwards, a layer sublimates entirely if $\delta I$ exceeds the layer ice mass, the layer is removed by setting its thickness to zero, and the ice mass is subtracted from $\delta I$ for the next layer. If the layer only partially sublimates, the layer thickness is reduced to $(1 - \delta I/I)D_{sn}$ and the loop through layers terminates.

\subsection{Snow density}

\subsubsection{Fixed density (option {\tt DENSTY 0})}

Snow has constant density $\rho_0$.

\subsubsection{Compaction with age (option {\tt DENSTY 1})}

The density of a snow layer with ice content $I$, liquid water content $W$ and thickness $D_{sn}$ at the beginning of a timestep is diagnosed as
\begin{equation}
\rho_s = \frac{I + W}{D_{sn}}.
\end{equation}
This is then incremented by
\begin{equation}
\delta\rho_s = (\rho_{\rm max} - \rho_s)(1 - e^{-\delta t/\tau_\rho}).
\end{equation}
Maximum density $\rho_{\rm max}$ has different values $\rho_{\rm cold}$ and $\rho_{\rm melt}$ for cold and melting snow. The thickness of the compacted layer at the end of the timestep is retrieved as
\begin{equation}
D_{sn} = \frac{I + W}{\rho_s}.
\end{equation}

\subsubsection{Compaction by overburden and thermal metamorphism (option {\tt DENSTY 2})}

Layer density is incremented by
\begin{equation}
\delta\rho_s = \rho_s\left\{\frac{gm}{\eta} + c_1 \exp\left[\frac{(T_{sn} - T_m)}{23.8} 
                                  - \max\left(\frac{\rho_s - 150}{21.7}, 0\right)\right]\right\}\delta t
\end{equation}
where
\begin{equation}
\eta = \eta_0\exp\left[-\frac{(T_{sn} - T_m)}{12.4} + \frac{\rho_s}{55.6}\right]
\end{equation}
and the snow mass overlying the middle of layer $n$ is
\begin{equation}
m_n = \sum_{j=1}^{n-1}(I_j + W_j) + 0.5(I_n + W_n).
\end{equation}

\subsection{Grain growth}

\subsubsection{Temperature metamorphism from JULES (option {\tt SGRAIN 1})}
The increment in grain radius in a snow layer over a timestep is $\delta r = g_r r^{-1}\delta t$ for grain area growth rate
\begin{equation}
g_r = 
\begin{cases}
    2\times10^{-13}\ {\rm m^2\ s^{-1}}  & T_{sn} = T_m \\
    2\times10^{-14}\ {\rm m^2\ s^{-1}}  & T_{sn} < T_m,r < 1.5 \times 10^{-4}{\rm m} \\
    7.3\times10^{-8} \exp(-4600/T_{sn}) & T_{sn} < T_m,r \geq 1.5 \times 10^{-4}{\rm m} \\
\end{cases}
\end{equation}

\subsubsection{Temperature gradient metamorphism from SNTHERM (option {\tt SGRAIN 1})}
The temperature gradient in a snow layer is calculated as the temperature difference between the top and bottom of the layer divided by the layer thickness. The temperature at the top of a snow layer is $T_s$ for the surface layer and 
\begin{equation}
T_{sn,n-1/2} = \frac{D_{sn,n-1}T_{sn,n} + D_{sn,n} T_{sn,n-1}}{D_{sn,n} + D_{sn,n-1}}
\end{equation}
for $n>1$. The temperature at the base of the snow is
\begin{equation}
T_{sn,n+1/2} = \frac{D_{sn,n}T_{sn,n} + \Delta z_{sl,1} T_{sl,1}}{D_{sn,n} + \Delta z_{sl,1}}
\end{equation}
for $n = N_{\rm snow}$. The vertical vapour flux in a layer is
\begin{equation}
q_v = 9.2\times 10^{-5}\left(\frac{T_{sn,n}}{T_m}\right)^6 \frac{\partial\rho_{\rm sat}}{\partial T}
                       \left(\frac{T_{sn,n-1/2} - T_{sn,n+1/2}}{D_{sn,n}} \right)
\end{equation}
with
\begin{equation}
\frac{\partial\rho_{\rm sat}}{\partial T} = \frac{e_0}{R_{\rm wat} T^2}
\left(\frac{L_s}{R_{\rm wat} T} - 1 \right)
\exp\left[\frac{L_s}{R_{\rm wat}}\left(\frac{1}{T_m} - \frac{1}{T}\right)\right].
\end{equation}
The grain area growth rate is
\begin{equation}
g_r = 
\begin{cases}
    1.25\times 10^{-7}\min(|q_v|, 10^{-6})  & \theta_w < 10^{-4}    \\
    10^{-12}\min(\theta_w+0.05, 0.14)       & \theta_w \geq 10^{-4} \\
\end{cases}
\end{equation}
where the volumetric liquid water content is
\begin{equation}
\theta_w = \frac{W}{\rho_{\rm wat}D_{sn}}.
\end{equation} 

\subsection{Adding and removing snow}

Snowfall and frost ($E<0$ for $T_s<T_m$) are added to layer 1 with fresh snow density $\rho_f$ and grain size $r_0$. Snow unloading from a forest canopy is added to snow on the ground with the existing bulk density and grain size.

All snow layers except the bottom layer have fixed (user specified or default) thicknesses. As snow depth changes, the thickness of the lowest layer changes until it falls below the fixed thickness or increases to more than twice the fixed thickness. The lowest layer is then combined with the layer above (decreasing) or split into two equal parts (increasing) if the maximum number of snow layers $N_{\rm smax}$ has not been reached. Ice mass, liquid water mass and internal energy contents of snow layers are stored before recalculating layer thicknesses and then redistributed according to conservation.

\subsection{Liquid water movement in snow}

\subsubsection{Free drainage (option {\tt HYDROL 0})}
Liquid water drains from snow immediately and is added to runoff.

\subsubsection{Bucket storage (option {\tt HYDROL 1})}
A snow layer with porosity
\begin{equation}
\phi = 1 - \frac{I}{\rho_{\rm ice}D_{sn}}
\end{equation}
can hold a maximum mass $W_{\rm max}=\rho_{\rm wat}\phi D_{sn} W_{\rm irr}$ of liquid water. If the liquid mass of a layer exceeds the maximum, the excess is passed downwards to the next layer.

\subsubsection{Gravitational drainage (option {\tt HYDROL 2})}
Liquid water in excess of layer porosities ($\theta_w>\phi$) is first added to runoff. Saturated hydraulic conductivity in a layer and downward water flux due to gravitational drainage at the bottom of the layer are then parametrized as
\begin{equation}
k_{{\rm sat},n} = 0.31\frac{\rho_{\rm wat}gr_n^2}{\mu_{\rm wat}}\exp\left(-7.8\frac{\rho_{s,n}}{\rho_{\rm wat}}\right)
\end{equation}
and
\begin{equation}
Q_{w,n+1/2} = k_{{\rm sat},n}\left(\frac{\theta_{w,n} - \theta_{r,n}}{\phi_n - \theta_{r,n}}\right)^3
\end{equation}
for irreducible water content $\theta_{r,n} = W_{\rm irr}\phi_n$. The conservation equation for changes in liquid water content is discretized by an implicit upwind scheme
\begin{equation}
\frac{\theta_{w,n} - \theta_{w,n}^{(0)}}{\delta t} = \frac{Q_{w,n-1/2} - Q_{w,n+1/2}}{D_{sn,n}}
\label{eq:water}
\end{equation}
with upper boundary condition $Q_{w,-1/2} = R_f/\rho_{\rm wat}$.

Each driving data timestep is divided into $N_{\tt shyd}$ substeps of length $\delta t_s = \delta t / N_{\tt shyd}$ to improve the numerical stability of solving equation (\ref{eq:water}). Layer liquid water contents at the end of each substep are found by the Newton-Raphson method, with increments in each iteration found by forward substitution. Zeros in the liquid mass balance residuals
\begin{equation}
r_n = \frac{\theta_{w,n} - \theta_{w,n}^{(0)}}{\delta t_s} + \frac{Q_{w,n-1/2} - Q_{w,n+1/2}}{D_{sn,n}}
\end{equation}
are sought by adding increments
\begin{equation}
\delta\theta_{w,n} = 
\begin{cases}
   -r_n/b_n                           & n = 1 \\
  -(a_n\delta\theta_{w,n-1}+r_n)/b_n  & 1 < n \leq N_{\rm snow}
\end{cases}
\end{equation}
and recalculating the water fluxes, with
\begin{equation}
a_n = -3\frac{k_{{\rm sat},n-1}}{D_{sn,n-1}}\frac{(\theta_{w,n-1} - \theta_{r,n-1})^2}{(\phi_{n-1} - \theta_{r,n-1})^3}
\end{equation}
and
\begin{equation}
b_n = \frac{1}{\delta t_s} + 3\frac{k_{{\rm sat},n}}{D_{sn,n}}\frac{(\theta_{w,n} - \theta_{r,n})^2}{(\phi_n - \theta_{r,n})^3}
\end{equation}


\subsubsection{Freezing of liquid water in snow}
The cold content of a snow layer is $C_c = C_{sn}(T_m - T_{sn})$. If this is greater than zero and there is liquid water in the layer, an amount $\delta I = \min(W,L_f^{-1}C_c)$ freezes and is added to the layer ice mass. An increment $L_f\delta I/C_{sn}$ is added to the layer temperature.


\section{Soil temperatures (subroutine {\tt SOIL})}

Heat flux at the soil surface is $G_{\rm soil}=G_s$ if there is no snow and $G_{\rm soil} = G_{N_{\rm snow}}$ at the base of the snow if there is snow. Timestep increments in soil layer temperatures are found by solving a tridiagonal matrix equation with thermal transmittance between soil layers ($U$), below-diagonal ($a$), diagonal ($b$) and above-diagonal ($c$) matrix elements, and right-hand side vector elements ($r$)
\begin{align}
U_n &= 2\left(\frac{\Delta z_{sl,n}}{\lambda_{sl,n}} + \frac{\Delta z_{sl,n+1}}{\lambda_{sl,n+1}}\right)^{-1} \\
b_n &= C_{sl,n} + U_n\delta t \\
c_n &= -U_n\delta t \\
r_n &=  [G_{\rm soil} + U_n(T_{sl,n+1} - T_{sl,n})]\delta t
\end{align}
for $n=1$,
\begin{align}
U_n &= 2\left(\frac{\Delta z_{sl,n}}{\lambda_{sl,n}} + \frac{\Delta z_{sl,n+1}}{\lambda_{sl,n+1}}\right)^{-1} \\
a_n &= -c_{n-1} \\
b_n &= C_{sl,n} + (U_{n-1} + U_n)\delta t \\
c_n &= -U_n\delta t \\
r_n &= U_{n-1}(T_{sl,n-1} - T_{sl,n})\delta t + U_n(T_{sl,n+1} - T_{sl,n})\delta t
\end{align}
for $n=2,...,N_{\rm soil}-1$, and
\begin{align}
U_n &= \frac{\lambda_{sl,n}}{\Delta z_{sl,n}} \\
a_n &= -c_{n-1} \\
b_n &= C_{sl,n} + (U_{n-1} + U_n)\delta t \\
r_n &= U_{n-1}(T_{sl,n-1} - T_{sl,n})\delta t
\end{align}
for $n=N_{\rm soil}$. This assumes that there is no heat flux at the base of the soil model.

\end{document}


